\chapter{Abstract}
% Approximately 300 words
Cycling is widely acknowledged as a valuable tool to reduce the dependance of transportation on fossil fuels and thus reduce its carbon cost. In addition to this it fosters healthier lifestyles, reduces sedentary habits and promotes cardiovascular fitness. However, the switch to cycling is hindered by safety fears, in particular regarding how drivers interact with cyclists when sharing the road. Current governmental policy places the responsibility on both cyclists and drivers for their own safety. Safety campaigns from responsible government agencies focus primarily on encouraging safer practices through personal protective equipment and behavioural changes. However, it is difficult to examine the effectiveness of these measures and recent crash statistics suggest little change in the number of cyclist fatalities. As the user with the greatest potential to cause harm, driver decision-making has an outsized impact on the safety of those cycling. This thesis involved the design of a decision neuroscience task in the Psychtoolbox plugin for MATLAB which could be used to examine that decision-making process. The task was shown to be capable of capturing the required patterns and richness of data used to fit the mechanistic models of decision making that characterize the field. This provides the groundwork for further investigation into the area, potentially allowing for the rigorous examination and mechanistic understanding of how drivers make decisions and how they could be influenced to make safer ones.
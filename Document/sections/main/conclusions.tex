\chapter[Conclusions]{Conclusions and Future Work}
% Gonna call this ~750
% Conclusions
The purpose of this project was the application of the tools of decision neuroscience to the ongoing, widespread road safety concerns faced by cyclists. In that vein, a paradigm was identified from observation where drivers would overtake cyclists for little or no gain. This paradigm was designed into a minimally-complex task capable of capturing a sufficient amount of the complexity of this decision to record meaningful data. The task was piloted and proven to be capable of gathering that meaningful behavioural data from subjects and that the data is consistent with the structure of that which is commonly seen in the results of decision neuroscience tasks. This task design is a strong foundation where upon further work will allow the creation of a mechanistic model and a greater understanding of the complex real-world situation it is designed to study.

% Future work
\section{Future Work}
% Future expanding the task#
The primary focus of future work should be two-fold. The first aim would be to further refine the task, address its flaws as detailed in the discussion, and further prove its viability as a method of capturing accurate behaviour during a driver/cyclist interaction. The second aim is the construction of a mechanistic model capable of recreating the results gathered from the improved task. This work will continue through the summer, building on this strong foundation to hopefully achieve a paradigm capable of sufficiently capturing the complexity of the situation to conclusively describe why it is that drivers make dangerous decisions when interacting with cyclists.
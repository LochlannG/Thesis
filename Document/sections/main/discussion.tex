\chapter{Discussion}
% Somewhere around 1250
\section{Response Times and Task Construction Verification}
The primary motivation behind this task construction was to create a minimally complex task capable of capturing the main contingencies and events of driving decisions when encountering a cyclist from behind. To this effect, the task was designed in such a fashion that it was expected that the results of the pilot trials would closely resemble those of other decision neuroscience tasks that had been examined throughout the project.
This was largely a success. The task that has been designed and piloted is capable of capturing meaningful behavioural data from subjects which reveals underlying mechanisms behind their decisions. The response time distributions seen in figures \ref{fig:RT_B} \& \ref{fig:RT_C} closely resemble the classic right-skewed distributions that are seen in these DN tasks. This is a promising result.

One observation worthy of discussion surrounding these reaction time distribution plots are the classification of outliers. Traditionally in DN tasks, response times significantly greater than the tails of the distributions would be considered to be 'false alarms'. However, due to the increased complexity of this task, most importantly the subjects control over their own speed, these decisions are valuable data points. They show that, if going sufficiently slow, subjects may have to wait a long time from stimulus presentation until it is appropriate to register their decision. However, the actual decision may have been made well in advance of the button press.

This is further reflected in figure \ref{fig:RT_SubjectMean} where the decision to overtake a cyclist has a significantly higher mean value than all others. This is likely due to the fact that subjects delay the recording of their decision until the cyclist has come closer to the camera. This is a unique feature of this task, it does not just require that subjects make a decision, but also that they decide when to make their decision. There are also differences evident between the mean value of slowing down when faced with a cyclist or a car. As subjects have been instructed to always slow down when a car is identified, it has an earlier mean response time. However, the cyclist stimulus involves making a decision and weighing up the costs, delaying the registration of that decision. This could be the result of an evidence accumulation process where subjects set their boundaries for the slow down decision to be higher in the case of the cyclist stimulus than the car stimulus.

\section{The Effect of View Distance \& Cyclist Appearance Distance on Response Time}
In figure \ref{fig:Proportion by View Dist} the proportion breakdown of decision made is plotted relative to the view distance. The result in this case was not as expected. The assumption was that in the case of a lower visibility, drivers would become far more cautious and thus make far fewer overtakes. However, this did not transpire as expected. Instead, they compensated for the lowered visibility by setting their speeds lower. This is discussed further in section \ref{secForRef}.

The cyclist appearance distance and its effect on response time and decision made can be seen in figure \ref{fig:CyclistPositionDist} a). It can be seen in this plot that subjects reacted in two distinct peaks when the cyclist appeared further away (40m \& 60m) and that the response times had greater variance when the cyclist appeared much closer to the camera (20m). Further breaking this down in figure \ref{fig:CyclistPositionDist}b) into the two choices made reveals the source of the double peak distribution. Subjects chose to slow down more quickly when the cyclist appeared further away. Conversely, for the same distances we see a wider peak, later in the overtake plot where subjects pressed the button at the appropriate time to overtake. This, as before, is evidence of the subject pressing the button to record a decision well after the actual decision has been made.

\section{Capturing Inter-Subject Variation}
As was identified in the results and what can clearly be seen in figure \ref{fig:Underlying_biases}, interpersonal differences dominated which choice was made. This is a fascinating result, it shows that subjects will decide on their 'overtaking strategy' almost independently of the actual situation being presented to them. Further analysis could reveal a basis for these tendancies, using the EMG recording would have been useful as, it may have been able to reveal an underlying tendancy in the activation of the dorsal interosseous muscle showing a consistent bias towards one of the options.

\section{Subject Risk Perception and Mitigation}
\label{secForRef}
One result of note was the statistical difference discovered between subject speed setting following an event. It was found that subjects would moderate their speed relative to the new view distance following a stimulus. However, upon pairwise comparison it was revealed that this effect was only present between the lowest (20m) level and the higher levels (40m, 60m). This would suggest that subjects evaluated their risk to be of an equivalent level at these two larger visibility values. Further investigation may reveal a boundary between these risk evaluation levels.

\section{Task Flaws, Limitations \& Edge Cases}
% The slow down bug, the edge cases resulting
% The lack of true continuous monitoring
In reality, driving is continuous in every sense. Decisions are made frequently and all aspects of the environment must be constantly monitored. This task did not contain a fully continuous framework. However, it effectively simulated the continuous monitoring environment while avoiding the complexities of its implementation. This gives the task experimental tractability.

There were some flaws in the task which became apparent only following the completion of the pilot trials. The clearest of these flaws was the ability for the subject to set their speed to be equal to that of the slowest object in their path. In this situation a stimulus may never appear on screen as their relative speed is $0km/h$. Fortunately this flaw is easily fixed, reducing the sensitivity of the speed setting will give a subject finer control over their speed. In addition to this subjects will be prevented from bringing themselves to this 'relative standstill'.

During the pilot trials there were some edge cases revealed that had not been identified during testing of the trial. One edge case of particular note was the interaction effect between the two measures of subject position; the pseudo-random generation of stimulus onset and start distance from camera values. This took the form of stimuli being drawn to the same position on screen unintentionally as there were no contingencies in the task construction to find situations where these two positions could overlap.

% \section{Setting Speed Based On View Distance}
% It was shown that subjects would moderate their speed based on the view distance shown to them in the task. This represents the identification of a certain degree of caution when visibility is restricted. However, the pairwise comparison revealed a more telling picture. Subjects would slow down at the lowest view distance but no statistically significant difference was found between the two higher distances. This could reveal an insight into the boundaries of this caution response. Those levels were set subjectively during task design, further examination could identify more characteristic values capable of identifying a more meaningful relationship between view distance and speed setting.

\section{Limitations of the EMG Recording Method}
The method of correlating the EMG traces to the task recording was ineffective. Whether this was a result of flaws in the EMG software, trigger sending, or the task code is unclear and determining this will be a not-insignificant portion of future work. What is clear however, is that more correlation time-points will be required. It was not sufficient to only send triggers to the EMG software during button presses, more regular and easily identifying triggers should be added to the task. Examples might include when a stimulus first appears, or at the beginning of a trial. This would allow the traces to be more easily dissected and correlated with each other.